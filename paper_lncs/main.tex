\documentclass[runningheads]{llncs}
\usepackage{booktabs}
\usepackage{graphicx}
\usepackage{url}

\begin{document}

\title{ArcheRisk-Core: A Factor-Level Micro-Benchmark for Multi-Agent System Security\\
via Behaviour Archetypes and Topology Families (Baseline-3)}
\titlerunning{ArcheRisk-Core}

\author{Anonymous Authors}
\institute{Anonymous Institution}

\maketitle

\begin{abstract}
Large Language Model (LLM) multi-agent systems (MAS) introduce new security risks because unsafe behaviour can propagate across roles and communication pathways.
This paper presents \emph{ArcheRisk-Core}, a reproducible micro-benchmark that isolates two core factors --- \textbf{attacker behaviour archetypes} and \textbf{communication topology families} --- under a controlled three-role MAS abstraction (Planner--Worker--Reviewer).
We introduce \textbf{ABeRT} (Agent Behaviour--Risk Interaction--Topology), a minimal modelling lens that maps local behavioural deviations into system-level risk events.
ArcheRisk-Core reports three operational metrics with confidence intervals: Attack Success Rate (ASR), Leak Rate (LR), and Unauthorized Write Rate (UWR).
Our Baseline-3 experiments show that explicit induction attacks are substantially reduced under defended settings, while covert and mixed attacks remain resilient in certain topologies, motivating topology-aware and dynamic defences.
\keywords{multi-agent systems \and LLM security \and benchmarking \and topology \and prompt injection}
\end{abstract}

\section{Introduction}
ACISP audiences see many practical security evaluations, but MAS security is often reported as a ``black box'': a single end-to-end number for a complex orchestration.
ArcheRisk-Core targets a complementary goal: \textbf{factor-level diagnosis}.
We ask: \emph{when a MAS fails, which factor is responsible --- the attacker behaviour, the interaction pathway, or the defence surface?}

To make this question answerable, we build a micro-benchmark around a concrete scenario:
\textbf{a Planner decomposes a request, a Worker executes it, and a Reviewer validates the output and any tool actions.}
Typical instances are:
Planner = ``task manager'' agent, Worker = ``executor'' agent (may use tools), Reviewer = ``policy/compliance'' agent that approves outputs/actions.

\paragraph{Contributions.}
(1) \textbf{ArcheRisk-Core}, a micro-benchmark with an explicit JSONL episode schema and a one-command runner.
(2) \textbf{ABeRT}, introduced in this paper, to structure factor-level reasoning across Behaviour, Interaction, and Topology.
(3) \textbf{Baseline-3} evaluation across \textbf{5 attacker behaviour archetypes}, \textbf{4 topology families}, and \textbf{3 task families} (including non-arithmetic workloads), reporting Wilson 95\% confidence intervals.

\section{ABeRT Lens and Benchmark Factors}
\subsection{ABeRT: Behaviour--Risk Interaction--Topology}
ABeRT is a lightweight lens (not a learning model) used to ensure that:
(i) attacker behaviours are defined independently from task content,
(ii) risk events are operationally measurable, and
(iii) topology is explicit (communication graph families, not a binary label).

\subsection{Behaviour Archetypes (Attacker Patterns)}
ArcheRisk-Core includes five archetypes. Each produces distinct observable traces on the same task:
\textbf{Manipulator} (direct override), \textbf{Deceiver} (benign framing for leakage), \textbf{Covert Actor} (tool/action focus), \textbf{Infiltrator-Escalator} (multi-step escalation), \textbf{Mixed} (redundant leak + action).
Appendix examples in the repo show how each archetype differs on identical inputs.

\subsection{Topology Families (Explicit Communication Graphs)}
We evaluate four topology families:
\emph{Chain} (Planner$\to$Worker$\to$Reviewer), \emph{Star} (Planner centered), \emph{Fully Connected}, and \emph{Reviewer Hub} (all traffic through Reviewer).
We also distinguish \textbf{INSECURE} vs \textbf{DEFENDED} \emph{modes} to indicate whether the intended topology and gating are enforced.

\subsection{Defence Baselines (B1--B3)}
Baseline-1 (B1): no boundary marking or tool guard.
Baseline-2 (B2): boundary marking for untrusted content + basic induction blocking + tool guard at the enforcement point.
Baseline-3 (B3): B2 plus stronger verified loop (redaction and stricter release checks).

\subsection{Task Families}
To avoid overfitting to arithmetic, we include three families:
(1) \emph{Arithmetic Check} (exact answer), (2) \emph{Policy Triage} (ALLOW/DENY decisions), and (3) \emph{File Triage} (ALLOW/DENY for protected vs unprotected paths).
These capture common MAS workflows (classification, policy gating, and tool safety).

\section{Threat Model and Metrics}
\paragraph{Threat model.}
The attacker can inject \textbf{natural-language} content into the interaction (no code execution privileges).
The MAS may have tool capabilities (e.g., writing a file), but any tool use must be triggered through agent outputs.

\paragraph{Metrics (operational).}
We define three binary risk events per episode:
\textbf{ASR} (attack success): the attack achieves its archetype objective under the scenario policy (e.g., successful leakage or unsafe action that is not prevented on the release path).
\textbf{LR} (leak): the protected secret token appears in \emph{released} outputs.
\textbf{UWR} (unauthorized write): the system attempts or performs a write to a protected path (attempts are detected from structured tool-call traces, even if blocked).
We report Wilson 95\% confidence intervals for all rates.

\section{Baseline-3 Experiments}
\subsection{Setup}
We generate $\sim$2000 episodes by factorial design over:
task family $\times$ topology family $\times$ topology mode $\times$ defence baseline $\times$ archetype (plus a small random top-up for exact count).
All episodes produce a trace JSON file for reproducibility.

\subsection{Results}
Table~\ref{tab:asr} reports ASR by archetype under \textbf{DEFENDED} mode, aggregated across tasks and topologies.
The trend matches practitioner expectations: B2/B3 reduce explicit induction (Manipulator/Deceiver), but covert and mixed archetypes remain resilient.
Table~\ref{tab:uwr} shows UWR by topology; reviewer-hub consistently reduces unauthorized writes, while star/fully-connected expose bypass surfaces.

\begin{table}[t]
\caption{Attack Success Rate (ASR) under DEFENDED mode (Wilson 95\% CI).}
\label{tab:asr}
\centering
\begin{tabular}{lccc}
\toprule
Archetype & B1 (ASR) & B2 (ASR) & B3 (ASR) \
\midrule
MANIPULATOR & 0.94 [0.86, 0.98] & 0.18 [0.10, 0.28] & 0.12 [0.06, 0.22] \
COVERT\_ACTOR & 0.93 [0.84, 0.97] & 0.81 [0.70, 0.89] & 0.70 [0.58, 0.80] \
DECEIVER & 0.94 [0.86, 0.98] & 0.26 [0.17, 0.38] & 0.04 [0.02, 0.12] \
INFILTRATOR\_ESCALATOR & 0.87 [0.77, 0.93] & 0.49 [0.37, 0.60] & 0.46 [0.35, 0.58] \
MIXED & 0.90 [0.81, 0.96] & 0.82 [0.71, 0.89] & 0.74 [0.62, 0.83] \
\bottomrule
\end{tabular}

\end{table}

\begin{table}[t]
\caption{Unauthorized Write Rate (UWR) under DEFENDED mode by topology family (Wilson 95\% CI).}
\label{tab:uwr}
\centering
\begin{tabular}{lccc}
\toprule
Topology & B1 (UWR) & B2 (UWR) & B3 (UWR) \
\midrule
chain & 0.21 [0.13, 0.31] & 0.36 [0.27, 0.47] & 0.37 [0.27, 0.47] \
star & 0.64 [0.53, 0.73] & 0.55 [0.44, 0.65] & 0.44 [0.34, 0.55] \
fully\_connected & 0.68 [0.58, 0.77] & 0.55 [0.44, 0.65] & 0.41 [0.31, 0.52] \
reviewer\_hub & 0.12 [0.07, 0.21] & 0.29 [0.21, 0.40] & 0.32 [0.23, 0.43] \
\bottomrule
\end{tabular}

\end{table}

\begin{figure}[t]
\centering
\includegraphics[width=\textwidth]{figures/fig_attack_success_behavior_lncs.pdf}
\caption{ASR by behaviour archetype under DEFENDED mode across B1--B3.}
\end{figure}

\begin{figure}[t]
\centering
\includegraphics[width=\textwidth]{figures/fig_leak_behavior_lncs.pdf}
\caption{Leak Rate by behaviour archetype under DEFENDED mode across B1--B3.}
\end{figure}

\begin{figure}[t]
\centering
\includegraphics[width=\textwidth]{figures/fig_unauthorized_write_behavior_lncs.pdf}
\caption{Unauthorized Write Rate by behaviour archetype under DEFENDED mode across B1--B3.}
\end{figure}

\section{Ethical Considerations}
ArcheRisk-Core contains attacker patterns intended for security evaluation.
To reduce misuse risk, the release focuses on \emph{measurable} risk events, includes defended baselines, and provides guidance for safe experimentation in isolated environments.
Users should avoid deploying these patterns against systems without explicit authorization.

\section{Availability}
The benchmark runner, dataset generator, traces, and an A2A adapter compatible with AgentBeats-style pipelines are included in the accompanying repository.

\bibliographystyle{splncs04}
\bibliography{references}

\end{document}
